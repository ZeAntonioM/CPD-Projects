\section{Results and Analysis}
We ran the algorithms three times 
and recorded the values of the chosen 
metrics. This section analyzes and 
compares the performance of the
algorithms based on the average 
values from these three runs.

\subsection{Basic multiplication and Line multiplication comparison}
Both \textbf{C++} and \textbf{Rust} are performance-oriented programming languages 
that provide developers with direct control over memory 
management and system interactions, facilitating the 
creation of highly optimized code. This is apparent 
when comparing the execution times of basic and line 
multiplication algorithms, where both C++ and Rust 
implementations demonstrate \textbf{similar performance}, as 
depicted in the \textit{(\hyperref[graph:BLG1]{Graph1}).}

Comparing the execution times of both algorithms, 
it's clear that the line multiplication algorithm 
\textbf{surpasses the basic multiplication algorithm in 
terms of speed} - \textit{(\hyperref[graph:BLG1]{Graph1}).} This can also be
evident in the \textit{(\hyperref[graph:BLG2]{Graph2})}, 
where the line multiplication algorithm demonstrates 
\textbf{fewer L1 and L2 cache misses} compared to its basic 
counterpart, resulting in a reduced execution time. 
Moreover, the correlation between FLOPS 
(Floating Point Operations Per Second) and cache misses 
is significant. The line multiplication algorithm 
consistently maintains higher FLOPS across varying 
matrix dimensions, indicative of its \textbf{efficient 
access to cache}. While there may be a 
slight decline in FLOPS as matrix dimensions increase, the 
line multiplication algorithm exhibits 
greater stability in its performance compared to the 
basic algorithm as depicted in \textit{(\hyperref[graph:BLG3]{Graph3})}.

\subsection{Block multiplication and Line multiplication comparison}
While performing the block multiplication algorithm, it was
possible to observe that for most of the matrix dimensions,
the block sizes of 512 and 256 outperformed the block size of 128
in terms of execution time eventhough the difference is
not quiet significant concluding that block size does not influence
the performance when comparing the performance of each block size.
 This is evident in the \textit{(\hyperref[graph:BLG4]{Graph4})} and 
\textit{(\hyperref[graph:BLG8]{Graph5})} as the execution time
for all block sizes are similar.

