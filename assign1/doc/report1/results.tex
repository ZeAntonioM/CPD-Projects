\section{Results and Analysis}
We ran the algorithms three times 
and recorded the values of the chosen 
metrics. This section analyzes and 
compares the performance of the
algorithms based on the average 
values from these three runs. The 
following graphs will illustrate the 
relationship between the input size 
(x-axis) and the corresponding metric 
values (y-axis).

\subsection{Execution Time and cache misses}
\subsubsection{Simple and line matrix multiplication}

\begin{tikzpicture}
    \begin{axis}[
        title={Execution Time Comparison},
        xlabel={Dimension},
        ylabel={Time (in s)},
        legend pos=north west,
        symbolic x coords={600, 1000, 1400, 1800, 2200, 2600, 3000},
        xtick=data,
        x tick label style={rotate=45,anchor=east},
        grid=major,
    ]
    \addplot[color=blue!65!white, mark=*] table[col sep=comma,x=Dimension,y=Time]{data/basicC++.csv};    \addplot[red, mark=square*, thick, dashed] table[col sep=comma,x=Dimension,y=Time]{data/basicRust.csv};
    \addplot table[col sep=comma,x=Dimension,y=Time]{data/lineC++.csv};
    \addplot table[col sep=comma,x=Dimension,y=Time]{data/lineRust.csv};
    \legend{c++ basic,Rust basic, c++ line,Rust line}
    \end{axis}
\end{tikzpicture}

\begin{tikzpicture}
    \begin{axis}[
        title={Cache Misses Comparison},
        xlabel={Dimension},
        ylabel={Cache miss (in units)},
        legend pos=north west,
        symbolic x coords={600, 1000, 1400, 1800, 2200, 2600, 3000},
        xtick=data,
        x tick label style={rotate=45,anchor=east},
        grid=major,
        yticklabel style={
            /pgf/number format/sci,
            /pgf/number format/precision=2
        },
    ]
    \addplot[blue, mark=*] table[col sep=comma,x=Dimension,y=L1]{data/lineL2.csv};
    \addplot[red, mark=square*, thick, dashed] table[col sep=comma,x=Dimension,y=L1]{data/lineL1.csv};
    \addplot table[col sep=comma,x=Dimension,y=L1]{data/basicL1.csv};
    \addplot table[col sep=comma,x=Dimension,y=L1]{data/basicL2.csv};
    \legend{L2 (line), L1 (line), L1 (basic), L2 (basic)}
    \end{axis}
\end{tikzpicture}