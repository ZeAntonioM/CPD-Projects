\section{Performance Metrics}

\subsection{Metrics}
To assess the performance of the C++ implementations, we 
utilized the PAPI (Performance API), providing us with comprehensive 
hardware performance monitoring. Our evaluation comprised the 
following key metrics:

\begin{itemize}
\item \textbf{Execution Time} - This metric quantifies the 
duration taken by the program to 
complete its execution. It serves as a fundamental 
benchmark for computing overall performance 
and facilitating comparative analysis.

\item \textbf{L1 and L2 Cache Misses} - These metrics 
quantify the number of cache misses occurring in the 
L1 and L2 caches, respectively. A cache miss arises 
when the requested data is not found in the cache and 
require retrieval from main memory. This metric 
provides valuable insights into the memory access patterns
of the program, as frequent cache misses can lead to 
increased execution times, which means that 
minimizing cache misses is crucial for optimizing 
program performance!

\item \textbf{MFlops} - This metric, derived from "\textbf{M}illion 
\textbf{F}loating-point \textbf{O}perations \textbf{p}er \textbf{S}econd", provides a measure of a 
computer's performance in executing floating-point arithmetic 
operations. It quantifies the rate at which a system can 
perform such operations and is calculated using the following formula:
    \begin{equation}
        MFlops = \frac{Number\ of\ Floating\ Point\ Operations}{Execution\ Time\ in\ Seconds} \times 10^{-6}
    \end{equation}
\end{itemize}

\subsection{Hardware used}
\begin{itemize}
    \item \textbf{Processor} - Intel Core i7-10700 CPU @ 2.90GHz 8-core (16 threads)
    \item \textbf{L1 Cache} - 512 KB
    \item \textbf{L2 Cache} - 2 MB
\end{itemize}
