\section{Results and Analysis}
We ran the algorithms three times
and recorded the values of the chosen
metrics. This section analyzes and
compares the performance of the
algorithms based on the average
values from these three runs.

\subsection{Basic multiplication and Line multiplication comparison}
Both \textbf{C++} and \textbf{Rust} are performance-oriented programming languages
that provide developers with direct control over memory
management and system interactions, facilitating the
creation of highly optimized code. This is apparent
when comparing the execution times of basic and line
multiplication algorithms, where both C++ and Rust
implementations demonstrate \textbf{similar performance}, as
depicted in the \textit{\hyperref[graph:BLG1]{Graph1}.}

Comparing the execution times of both algorithms,
it's clear that the line multiplication algorithm
\textbf{surpasses} the basic multiplication algorithm in
terms of speed - \textit{\hyperref[graph:BLG1]{Graph1}.} This can also be
evident in the \textit{\hyperref[graph:BLG2]{Graph2}},
where the line multiplication algorithm demonstrates
\textbf{fewer L1 and L2 cache misses} compared to its basic
counterpart, resulting in a reduced execution time.
The line multiplication algorithm's cache efficiency excels from its
\textbf{sequential and contiguous} memory access. This pattern optimally
utilizes cache by accessing adjacent data, aligning well with
cache loading mechanisms. In contrast, the basic algorithm's
non-sequential access leads to \textbf{increased cache misses} and
consequently lower FLOPS, impeding performance.
This can be observed in the \textit{\hyperref[graph:BLG3]{Graph3}}
as the line multiplication algorithm
consistently maintains higher FLOPS across varying
matrix dimensions, indicative of its \textbf{efficient
    access to cache}. While there may be a
slight decline in FLOPS as matrix dimensions increase, the
line multiplication algorithm exhibits
greater stability in its performance compared to the
basic algorithm as predicted.

\subsection{Block multiplication and Line multiplication comparison}
During the testing of the block multiplication algorithm, it
became evident that this approach \textbf{outpaces} both the line multiplication
algorithm and the basic multiplication algorithm, as anticipated.
This superiority is clearly seen in the \textit{\hyperref[graph:BLG8]{Graph5}}, where the
block multiplication method showcases a substantial \textbf{decrease in
    execution time} compared to the line multiplication algorithm.
Furthermore, as shown by the \textit{\hyperref[graph:BLG4]{Graph4}}, while there
may be slight fluctuations in execution time, it's notable that
different block sizes exhibit similar performance, although using blocks of
size 512 appears to give optimal results.

As previously discussed, faster execution times often correlate with fewer cache
misses and higher FLOPS, implying that the
block multiplication algorithm would \textbf{likely outperform} the line multiplication
algorithm in these metrics. This expectation is partially supported by the data in
the \textit{\hyperref[graph:BLG6]{Graph6}}, where the block multiplication algorithm
exhibits \textbf{lower L1 cache misses} compared to the line multiplication algorithm.
However, upon examining the \textit{\hyperref[graph:BLG7]{Graph7}},
it becomes apparent that although the difference is not that much,
the line multiplication algorithm demonstrates \textbf{better performance in terms of L2
    cache misses}. This observation contrasts with the initial hypothesis. To
understand this apparent contradiction, it's crucial to recognize the \textbf{significant
    speed disparity} between \textbf{L1 and L2 caches}. The block multiplication algorithm's
strategy of breaking matrices into smaller blocks enhances spatial locality,
reducing L1 cache misses by facilitating \textbf{more efficient data access and reuse}.
Consequently, data required for calculations is \textbf{often readily available in the
    faster L1 cache}, minimizing the time required to fetch data from slower memory.
However, due to the algorithm's utilization of larger amounts of data overall,
it may exceed the capacity of the L2 cache, resulting in more misses at this
level. Despite these increased L2 cache misses, the block multiplication
algorithm remains faster overall due to its efficient use of the L1 cache.

When comparing the FLOPS of the block multiplication algorithm to
the line multiplication algorithm, we anticipated \textbf{superior performance
    from the block algorithm}. Additionally, we expected that varying block
sizes would demonstrate consistency in FLOPS across different matrices sizes,
with a block size of 512 being the most effective, followed by 256 and 128.
These expectations were \textbf{largely
    verified} by the data presented in the \textit{\hyperref[graph:BLG9]{Graph8}},
except for a \textbf{notable anomaly}:
when using block sizes of 256 and 512 on a matrix with dimensions of 8192,
there was a \textbf{significant decline in FLOPS}, contrasting with the consistent
performance observed across other matrix dimensions. Despite in-depth analysis,
we couldn't identify a noticeable pattern in the data to explain this phenomenon,
particularly given that even on matrices with higher dimensions, the FLOPS remained
consistent.

\newpage

\subsection{Line multiplication algorithm - Parallel versions}
This section explores the performance gains achieved through two distinct
approaches for parallelizing a nested loop structure using OpenMP's \textbf{\#pragma omp parallel
    for} instruction.

\hspace*{1cm}

\begin{lstlisting}[language=C++, caption=Parallel 1]
    #pragma omp parallel for
    for (int i=0; i<n; i++) {
        for (int k=0; k<n; k++) {
            for (int j=0; j<n; j++) {
                // computations here
            }
        }
    }
\end{lstlisting}

\begin{itemize}
    \item Distributes iterations of the \textbf{outer loop (i)} among multiple threads for concurrent execution
    \item \textbf{The inner loops (k and j)} are executed sequentially within each thread
\end{itemize}

\hspace*{1cm}

\begin{lstlisting}[language=C++, caption=Parallel 2]
    #pragma omp parallel
    for (int i=0; i<n; i++) {
        for (int k=0; k<n; k++) {
            #pragma omp for
            for (int j=0; j<n; j++) {
                // computations here
            }
        }
    }
\end{lstlisting}

\begin{itemize}
    \item Distributes iterations of the \textbf{innermost loop (j)} among multiple threads for concurrent execution
    \item \textbf{The outer loops (i and k)} are executed sequentially within each thread
\end{itemize}

\hspace*{1cm}

While both parallelization strategies managed to surpass the performance of the
sequential version, it was evident that the first strategy outperformed the second
by approximately 4.75 times, and exceeded the sequential version by a factor of 5.6,
as shown in the \textit{\hyperref[graph:BLG5]{Graph9}}. Despite this significant gap, the second strategy still 
demonstrated a slight improvement over the sequential version, achieving a speedup 
of 1.18.

As highlighted in the \textit{\hyperref[sec:metrics]{metrics section}}, the efficiency of parallelization strategies 
can be quantified using the \textit{\hyperref[eq:efficiency]{efficiency equation}}. 
As expected, the strategy that achieved the highest speedup also exhibited the highest 
efficiency - \textit{\hyperref[graph:BLG11]{Graph11}}. This efficiency metric serves as a 
validation of the achieved FLOPS by the parallelized versions, with the most 
efficient strategy naturally capable of delivering higher FLOPS, as illustrated 
in the \textit{\hyperref[graph:BLG10]{Graph10}}.