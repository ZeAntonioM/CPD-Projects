\section{Algorithms}
In order to understand the influence of different algorithms on
processor performance during matrix multiplication, we implemented
three algorithms that vary primarily in their memory access patterns.
To analyze the impact of the programming languages on the performance,
we implemented the algorithms in \textbf{C++} and \textbf{Rust}. This selection allows us to compare two widely used programming languages with
distinct memory management approaches. Both
languages offer high levels of control over memory management, making
them suitable for investigating how memory access patterns affect performance.


\begin{itemize}
    \item \textbf{Basic matrix multiplication algorithm} - implemented in C++ and Rust
    \item \textbf{Line matrix multiplication algorithm} - implemented in C++ and Rust
    \item \textbf{Block matrix multiplication algorithm} - implemented in C++
\end{itemize}

\subsection{Basic matrix multiplication algorithm}

This algorithm implements a basic matrix multiplication approach,
where each element in the resulting matrix is computed by
multiplying one row of the first matrix by one
column of the second. As a result, the algorithm demonstrates a
computational time complexity of $O(n^3)$ and a space complexity
of $O(n^2)$, owing to its composition of three nested loops. Here,
$n$ represents the size of the input matrices.

The resulting matrix can be calculated using the following equation:

\begin{equation}
    C[i][j] = \sum_{k=0}^{n-1} A[i][k] \cdot B[k][j]
\end{equation}

where $C$ is the resulting matrix, $A$ and $B$ are the input matrices, and $n$ is the size of the matrices.


\subsection{Line matrix multiplication algorithm}

Instead of directly multiplying a row of the
first matrix with a column of the second, this algorithm
computes the product of a row of the first
matrix with each column of the second. Despite maintaining the same time and space complexity as the algorithm before, this algorithm's
efficient cache usage often reduces cache misses and improves
execution time, particularly for large matrices.


\subsection{Block matrix multiplication algorithm}

The Block Matrix Multiplication algorithm is designed
to further optimize memory access patterns by dividing
the matrices into smaller blocks. This strategy enhances
cache utilization and reduces cache misses compared to the
Line Matrix Multiplication approach. The algorithm operates by
multiplying submatrices
within these blocks, thereby minimizing the number of accesses to slower memory
and exploiting spatial locality. Although
sharing identical space and time complexities with the preceding
algorithms, block matrix multiplication excels in performance,
particularly with large matrices. This superiority derives from its
capability to partition matrices into smaller blocks, effectively
decreasing the incidence of cache misses.